\beginsong{Ballade der Mittelmäßigkeit}[
    txt={Jakob Hoffmann}, 
    txtjahr={2019},
    mel={Georges Brassens (Les Copains d'abord)},
    meljahr={1964},
    index={Es ist halb acht, es ist schon hell},
]
\beginverse*
\nolyrics Intro: \[F] \[G7] \[B&] \[A-A7] \[Dm] \[G7-C7-F] \[A7]
\endverse

\beginverse
% Es ist halb \[D]acht, es ist schon hell,
% der Wecker klingelt und ganz schnell
% ist Gerd aus \[E7]seinem Bett heraus, er freut sich und weiß:
% Heut geht es \[G]los, heut fahr'n wir weg,
% zu diesem \[F#]wunderbaren \[F#7]Fleck,
% wo sind die \[Hm]Socken, wo Be\[E7]steck? Ach, ich \[A]freu mich \[D]so!
\endverse
\includegraphics[draft=false, page=1]{Noten/BalladeVomMittelmass.pdf}

\beginverse\memorize
Gerd hat ne \[D]Freundin, Amelie, sie spielt Klavier wie ein Genie,
das ist fan\[E7]tastisch, denkt er sich, Gerd hört gerne zu.

Es spielt \[G]Gitarre, leider nur reicht's bloß für \[F#]A-Moll und G-\[F#7]Dur,
für die \[Hm]Band war das nicht ge\[E7]nug, leider \[A]nicht ge\[D]nug.
\endverse

\beginverse
Der Nachbars^junge heißt Jean-Claude, er spricht französisch wie ein Gott,
liest Baude^laire im Original, Gerd findet's famos. 

Im Urlaub ^in der Bäckerei Croissants be^stellen sogar ^deux,
so^weit recht des Gerds La^tein, weiter ^aber ^nicht.
\endverse

\beginverse
In seine ^Klasse geht Chantal, sie spielt perfekt den Tennisball,
sie hat viel ^Ehrgeiz und Talent - bei Gerd ist das so:

Das Fußball^spielen macht im Spaß, doch ist er ^dabei nicht das ^As,
viel ^eher Karo-^Sieben oder^ vielleicht ^Acht.
\endverse

\beginverse
Mit seinem ^Kumpel Ferdinand zockt er gern Minecraft stundenlang
und Ferdi ^kann noch mehr, er kann programmieren.

Geht in den ^Junior Science Club, das ist für ^Gerd dann doch zu ^hart.
Er holt ^sich zwei Kugeln ^Eis, setzt sich ^in den ^Park. \[C]
\endverse

\beginverse*
\nolyrics Zwischenspiel: \[F] \[G7] \[B&] \[A-A7] \[Dm] \[G7-C7-F] \[A7]
\endverse

\beginverse
Gerd ist ein ^richtig netter Typ, dem alles mittelprächtig liegt, 
er mag die ^Menschen und das Neue Pop-Rock-Chansons.

Er macht die ^Rucksackklappe zu, er bindet ^seine Wander^schuh',
er ^könnte schrei'n vor ^Glück, endlich ^geht es ^los.
\endverse

\beginverse*
\nolyrics Ende: \[D] \[E7] \[G] \[F#-F#7] \[Hm] \[E7-A-D]
\endverse


\endsong
