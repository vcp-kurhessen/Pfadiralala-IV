\beginsong{So ist Versöhnung}[txt={Jürgen Werth}, mel={Johannes Nitsch}, index={Wie ein Fest nach langer Trauer}]


\beginverse\memorize
Wie ein \[Em]Fest nach langer \[Hm]Trauer, wie ein \[C]Feuer in der \[Em]Nacht,
ein off'nes Tor in einer \[Hm]Mauer, für die \[C]Sonne \[D]aufge\[G]macht.
Wie ein \[Am]Brief nach langem \[D]Schweigen, wie ein \[G]unverhoffter \[C]Gruß,
wie ein \[Am]Blatt an toten \[Hm]Zweigen, ein ich-\[C]mag-dich-\[D]trotzdem \[Em]Kuss.
\endverse

\beginchorus
So ist Ver\[D]söhn\[G]ung. So muss der \[D]wahre Friede \[G]sein. 
So ist Ver\[D]söhn\[Em]ung. So ist Ver\[C]geben und Verz\[Em]eih'n. 
\endchorus

\beginverse
Wie ein ^Regen in der ^Wüste, frischer ^Tau auf dürrem ^Land,
Heimatklänge für Ver^misste, alte Fein^de, Hand ^in ^Hand.
Wie ein ^Schlüssel im Ge^fängnis, wie in ^Seenot - Land in ^Sicht.
Wie ein ^Weg aus der Be^drängnis wie ein ^strahlen^des Ge^sicht.
\endverse

\printchorus

\beginverse
Wie ein ^Wort von toten ^Lippen, wie ein ^Blick, der Hoffnung ^weckt,
wie ein Licht auf steilen ^Klippen, wie ein ^Erdteil, neu ^ent^deckt.
Wie der ^Frühling, wie der ^Morgen, wie ein ^Lied, wie ein Ge^dicht,
wie das ^Leben, wie die ^Liebe, wie Gott ^selbst, das ^wahre ^Licht.
\endverse 

\printchorus


\endsong

\begin{intersong}

\end{intersong}