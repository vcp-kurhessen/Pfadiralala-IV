\beginsong{Unter dem Pflaster}[
    wuw={Angi Domdey (Schneewittchen)},
	jahr={1976}, 
]

\beginverse
\endverse 
\includegraphics[draft=false, width=1\textwidth]{Noten/UnterDemPflaster.pdf}

\beginverse
\[G]Komm lass dir \[Em]nicht erzählen, \[Am]was du zu \[D]lassen hast.
\[G]Du kannst doch \[Em]selber wählen,\[Am] nur langsam, \[D]keine Hast.
\endverse

\beginchorus
\[Gm]Unter dem \[F]Pflaster, ja \[Gm]da \[F]liegt der \[Gm]Strand,
\[Gm]komm reiß auch \[F]du ein paar \[Gm]Steine \[F]aus dem \[Gm]Sand.
\endchorus

\beginverse
^Zieh' die Schuhe ^aus,^ die schon so lang dich ^drücken.
^Lieber barfuß^lauf,^ aber nicht auf ^ihren Krücken.
\endverse

\printchorus

\beginverse
^Dreh' dich und ^tanz, ^dann könn'n sie ^dich nicht packen.
^ Verscheuch' sie ^ganz ^mit einem ^lauten Lachen.
\endverse

\printchorus

\beginverse
^ Deine größte Kraft^ ^ist deine Phanta^sie.
^ Wirf die Ketten ^weg ^und schmeiß sie ^gegen die,
\[G] die mit ihrer \[Em]Macht\[Am] deine Kräfte \[D]brechen wollen.
\endverse

\printchorus

\endsong

\beginscripture{- Angi Domdey}
Es ist ein Lied für die Phantasie und gegen die festgefahrenen, verhärteten Strukturen unserer Gesellschaft, gegen den harten Beton unserer Städte und die Versteinerung unserer Gedanken und Taten. Es ist ein Emanzipationslied, nicht nur für Frauen. Die Steine sollen nicht zum Werfen benutzt werden, sondern der Sand unter den Steinen zum Tanzen frei gelegt werden.
\endscripture
