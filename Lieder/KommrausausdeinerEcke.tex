\beginsong{Komm raus aus deiner Ecke}[
    index={Die Morgensonne glitzert auf dem See}, 
    jahr={2005},
    lager={VCP Bundeslager 2006: 100 PRO. Die Welt gestalten},
]

\beginverse
Die \[E]Morgensonne \[E7]glitzert auf \[A]dem See, 
\[G]in Wipfeln rauscht der Wind, \[D]alles riecht so frisch.
Ich \[E]fühl' mich frei, wenn \[E7]ich ins Leben \[A]geh'.
\[G]Gehöre ich dazu? \[D]Wie kann das besteh'n? 
\[G]Hörst du den Wind?
\[A]Eine neue Zeit be\[C]ginnt.
Und wenn \[D]du's nicht tust, macht's keiner.
\endverse

\beginchorus
Komm \[A]raus aus deiner \[E]Ecke, klatsch \[D]mit uns in die \[F]Hand!
\[G]Hundert \[A]pro den Frieden \[E]bauen mit \[F]Herz und mit Ver\[G]stand.
Komm \[A]her und zieh mit \[E]uns den \[D]Karren aus dem \[F]Sand!
\[G]Hundert \[A]pro die Welt ge\[E]stalten, bist \[F]du dazu be\[G]reit? Yeah! 
\endchorus

\beginverse
Das \[E]Lagerfeuer \[E7]lädt zum Singen \[A]ein,
\[G]ich fühle mich geborgen, \[D]hier bin ich, wer ich bin. 
Komm \[E]setz dich her, du \[E7]bist hier nicht al\[A]lein.
\[G]Gehöre ich dazu? \[D]Wie kann das entsteh'n?
\[G]Hörst du den Wind?
\[A]Eine neue Zeit be\[C]ginnt. 
Und wenn \[D]du's nicht tust, macht's keiner.
\endverse

\renewcommand{\everychorus}{\textnote{\bf Refrain (wdh.)}}
\beginchorus
\endchorus

\beginverse
Das ^Leben wirft so ^viele Fragen ^auf.
^Wir sind in der Kapelle ^und sprechen ein Gebet. 
Du ^schenkst Vertrauen ^und gibst festen ^Halt.
^Gehöre ich dazu? ^Lass sowas entsteh'n.
^Hörst du den Wind?  
^Eine neue Zeit be^ginnt.  
Und wenn ^du's nicht tust, macht's keiner.  
\endverse

\beginchorus
\endchorus

\beginverse*
\[Am]Das sind uns're Fragen, die \[Dm]Antwort fällt so schwer.  
Wer \[F]kann sie mit uns \[G]suchen? Bist \[Am]du dazu be\[E7]reit?
\[Am]Wie wird uns're Zukunft? Ge\[Dm]stalten wir sie fair?  
\[F]Das sind uns're \[G]Fragen. Die \[Am]Antwort fällt so \[E]schwer,  
die fällt so \[E7]schwer.
\endverse

\beginchorus
\endchorus

\endsong

\beginscripture{}
Lagerlied des VCP-Bundeslagers 2006 in Großzerlang im Norden Brandenburgs.
\endscripture
