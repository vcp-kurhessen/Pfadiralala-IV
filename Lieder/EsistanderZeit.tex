\beginsong{Es ist an der Zeit}[wuw={Eric Bogle, 1976 (freie deutsche Nachdichtung durch Hannes Wader, 1980)}, pfii={86}, pfiii={43}, gruen={199}, biest={644}, index={Weit in der Champagne}]

\beginverse
\endverse
\centering\includegraphics[width=1\textwidth]{Noten/Lied036.pdf}	

%\centering\includegraphics[width=1\textwidth]{Noten/Lied036_1.pdf}	

\beginverse\memorize
\[G]Hast du, toter Sol\[Em]dat, mal ein \[C]Mädchen ge\[Am]liebt?
Sicher \[D]nicht, denn nur dort, wo es \[G]Frie\[C]den \[G]gibt,
\[G]können Zärtlichkeit \[Em]und Ver\[C]trauen ge\[Am]deih'n.
Warst \[D]Soldat, um zu sterben, nicht \[G]um jung \[D]zu \[G]sein.
Viel\[G]leicht dachtest \[Em]du dir: Ich \[Am]falle schon bald,
nehme \[D7]mir mein Vergnügen, wie es \[G]kommt, mit Ge\[D7]walt.
Dazu \[G]warst du ent\[Em]schlossen, hast \[Am]dich aber dann
vor \[D]dir selber geschämt und es doch \[G]nie \[D]ge\[G]tan.
\endverse

\beginchorus
Ja, auch \[D]dich haben sie schon \[C]genauso \[G]belogen,
so, wie \[D]sie es mit uns heute \[C]immer noch \[G]tun,
und du \[C]hast ihnen alles \[D]gegeben:
Deine \[G]Kraft, deine \[C]Jugend, dein \[D]Le\[G]ben.
\endchorus

\beginverse
^Soldat, gingst du ^gläubig und ^gern in den ^Tod?
Oder ^hast du verzweifelt, ver^bittert, ^ver^roht
^deinen wirklichen ^Feind nicht er^kannt bis zum ^Schluss?
^Ich hoffe, es traf dich ein sau^be^rer ^Schuss.
Oder ^hat ein Ge^schoss dir die ^Glieder zerfetzt?
Hast du ^nach deiner Mutter ge^schrien bis zu^letzt?
Bist du ^auf deinen ^Beinstümpfen ^weitergerannt?
Und ^dein Grab, ^birgt es mehr als ein Bein, ^eine ^Hand?
\endverse
%\renewcommand{\everychorus}{\textnote{\bf Refrain (wdh.)}}

\beginchorus
Ja, auch \[D]dich haben sie schon \[C]genauso \[G]belogen,
so, wie \[D]sie es mit uns heute \[C]immer noch \[G]tun,
und du \[C]hast ihnen alles \[D]gegeben:
Deine \[G]Kraft, deine \[C]Jugend, dein \[D]Le\[G]ben.
\endchorus

\beginverse
^Es blieb nur das ^Kreuz als ^einzige ^Spur
von ^deinem Leben, doch ^hör' mei^nen ^Schwur
^für den Frieden zu ^kämpfen und ^wachsam zu ^sein.
^Fällt die Menschheit noch einmal auf Lü^gen ^he^rein,
dann ^kann es ge^scheh'n, dass bald ^niemand mehr lebt,
niemand, ^der die Milliarden von ^Toten be^gräbt.
Doch längst ^finden sich ^mehr und mehr ^Menschen bereit,
die^sen Krieg zu verhindern, es ist ^an ^der ^Zeit.
\endverse

\beginchorus
Ja, auch \[D]dich haben sie schon \[C]genauso \[G]belogen,
so, wie \[D]sie es mit uns heute \[C]immer noch \[G]tun,
und du \[C]hast ihnen alles \[D]gegeben:
Deine \[G]Kraft, deine \[C]Jugend, dein \[D]Le\[G]ben.
\endchorus

\endsong

\beginscripture{}
Das Lied ist die deutsche Variante von ''The green fields of France'' von Eric Bogle und wurde in dieser Version zu einer Hymne der Friedensbewegung in den 1980er Jahren.
\endscripture
