\beginsong{Aufstehn'n, aufeinander zugeh'n}[
    wuw={Clemens Bittlinger}, 
    index={Wir wollen aufsteh'n},
]

%\includegraphics[draft=false, width=1\textwidth]{Noten/Lied106.pdf}

\beginchorus
Wir wollen \[D]aufsteh'n, aufeinander \[A]zugeh'n, voneinander \[Hm]lernen miteinander \[F#m]umzu\[A]geh'n
\[D]Aufsteh'n, aufeinander \[A]zugeh'n und uns nicht ent\[Hm]fernen wenn wir etwas \[F#m]nicht ver\[A]steh'n 
\endchorus

\beginverse
\[G]Viel zu \[A]lang schon \[G]rumge\[A]legen, \[Hm]viel zu \[F#m]viel schon \[A]diskutiert.
\[G]Es wird \[A]Zeit sich \[G]zu be\[A]wegen; \[F#m]höchste Zeit, dass \[G]was pas\[A]siert.
\endverse

\beginchorus
\endchorus

\beginverse
^Jeder ^hat was ^einzu^bringen, ^diese ^Vielfalt: ^Wunderbar!
^Neue ^Lieder ^woll'n wir ^singen, ^neue Texte ^laut und ^klar.
\endverse

\beginchorus
\endchorus

\beginverse
^Dass aus ^Fremden ^Nachbarn ^werden, ^das ge^schieht nicht ^von allein,
^Dass aus ^Nachbarn ^Freunde ^werden, ^dafür setzen ^wir uns ^ein.
\endverse

\beginverse*
{\nolyrics Zwischenspiel: \[D] \[A] \[G] \[D]}

Outro: \lrep \[D]Dab dab dabedu \[A]dada, \[G]dab dab dabedu \[D]da \rrep
\endverse

\beginchorus
\endchorus

\endsong
