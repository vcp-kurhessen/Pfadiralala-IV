\beginsong{Lagerboogie}[pfii={5}, index={Ich kenn' die Tante Frieda}]

\beginverse
Ich \[D]kenn' die Tante Frieda, die wohnt im alten \[A]Haus,
und diese Tante Frieda, die kenn' ich ganz \[D]genau.
Mit einem Eimer Wasser putzt sie das ganze \[A]Haus
und was davon noch übrig bleibt, da macht sie Kaffee \[D]draus.
\endverse

\beginchorus
\lrep Jajaja tschu tschu, der Lagerboogie ist unser \[A]Boogie-Woogie. 
Tschu tschu tschu, die Zeit vergeht im \[D]Nu. \rrep
\endchorus

%\centering\includegraphics[width=1\textwidth]{Noten/Lied060.pdf}	

\beginverse
Der \[D]Lehrer in der Schule macht seinen Kindern \[A]klar,
dass Adam und dass Eva aus einer Rippe \[D]war.
Da meldet sich das Fritzchen: ''Herr Lehrer, au, au, \[A]au!
Mir tut die rechte Rippe weh, ich glaub', ich krieg' 'ne \[D]Frau.''
\endverse

\renewcommand{\everychorus}{\textnote{\bf Refrain (wdh.)}}
\beginchorus
\endchorus

\beginverse
Die ^Mutter liegt im Krankenhaus, der Vater im Sing-^Sing.
Die Oma geht mit Negern aus, die Kinder tanzen ^Swing.
Die Tante hat ein Kind gekriegt, man weiß noch nicht von ^wem.
Der Nachbar hat 'nen Schäferhund, vielleicht ist es von ^dem. 
\endverse

\beginchorus
\endchorus

\beginverse
Ja ^neulich in Kentucky, da ist mal was pas^siert,
da hab'n sie 'nen Verbrecher zum Galgenstrick ge^führt.
Nach etwa einer Stunde, da rief doch dieser ^Schuft:
''Zieht doch nicht so feste, ich krieg' ja keine ^Luft!''
\endverse

\beginchorus
\endchorus

\beginverse
Herr ^Meier kam geflogen auf einem Fass Ben^zin,
da dachten die Franzosen, es wär' ein Zeppe^lin.
Sie luden die Kanonen mit Sauerkraut und ^Speck
und schossen dem Herrn Meier die Unterhosen ^weg.
\endverse

\beginchorus
\endchorus

\endsong

\beginscripture{}
Sing-Sing = Hochsicherheitsgefängnis in der Nähe von New York City.
\endscripture
