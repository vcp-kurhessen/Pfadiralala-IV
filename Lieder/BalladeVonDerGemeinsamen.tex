\beginsong{Ballade von der gemeinsamen Zeit}[wuw={Milch \& Blut}, alb={Frag nicht!}, jahr={2004}]

\beginverse
\[Dm]Zähle doch nicht unsere \[Gm]Stunden und weine doch nicht, wenn du \[Dm]gehst. 
Du ver\[Dm]gießt doch \[C]auch keine \[B&]Tränen,\[A] \[B&]wenn der \[C]Wind mal nicht \[F]weht. 
So \[G#]frier' ich auch nicht in der \[Gm]Nacht, wenn der Mond am Himmel ver\[F]rät, 
dass die \[F]Sonne ihr \[D#]Licht nun wo\[C#]anders aus\[C]trägt. 

\[Dm]Halte mich in deinen \[Gm]Armen und lass uns gehen ein \[Dm]Stück. 
\[Dm]Andere \[C]machen das \[B&]anders,\[A] doch was \[B&]wissen denn \[C]die schon von \[F]Glück? 
Was \[G#]wissen denn die schon von \[Gm]Abschied und ist's nur ein Abschied von \[F]Zeit, 
so \[F]hab' ich doch \[D#]einen \[C#]Zeitver\[C]treib.
\endverse

\beginchorus
\[Dm]Heute sä' ich, morgen mäh' ich,\[Gm] übermorgen back' ich Brot, 
\[C]press' den Saft aus Südhangreben, \[F]dieser Wein wird \[A]süß und rot. 
\[Dm]Bau ein Haus aus Wegrandsteinen,\[Gm] pflanze Rosen, roten Mohn, 
\[C]lern' das schöne Spiel der Geige, \[F]kauf' dir ein Ban\[A]doneon. 
\[Dm]Hack' das Holz, heiz' die Stube,\[Gm] nehm' ein Bad mit Elixier, 
\[C]reiß' die Blätter vom Kalender \[F]und dann bist du \[A]wieder hier. 
{\nolyrics Zwischenspiel: \[Dm Gm C F A Dm]}
\endchorus

\beginverse
So ^kamst du zurück eines ^Tages, dein Koffer verschwand unter'm ^Bett, 
jetzt ^liegst du ^in meinen ^Armen,^ doch ^weiß ich, du ^gehst wieder ^weg. 
Noch ^halten wir unsere ^Wärme, noch lächelt dein Ge^sicht, 
noch ^drücken die ^Koffer ^unter uns ^nicht. 
\newpage
Dann ^sagst du, du hast noch zwei ^Stunden, dann ruft dich wieder die ^Pflicht. 
Wir ^haben 'ne ^Art ge^funden,^ dass ^uns das ^Herz nicht zer^bricht. 
Unser ^Gang endet wieder am ^Bahnsteig, ich seh' zu wie der Zug sich ent^fernt, 
hör ^zu, ich ^hab dieses ^Lied ge^lernt: 
\endverse

\printchorus

\beginverse
^Hat man uns denn so er^zogen, was hat uns soweit ge^bracht, 
dass ^dieses ^dumme ^Leben^ uns ^hindert an ^unserer ^Pracht? 
Uns ^hindert an unserer ^Nähe, denn die Liebe verhindert's ja ^nicht, 
wie die ^Traurig^keit, wenn der ^Morgen an^bricht. 


Was ^soll das viele Ge^renne, und was sagt mir dies klagende ^Lied, 
es ^sagt mir, ^dass sich nichts ^ändert,^ ^wenn keine ^Änd'rung ge^schieht. 
Wir ^haben nur ein kurzes ^Leben, dann sind wir wieder al^lein, ja
so ^könnt es ^jetzt doch mal ^andersrum ^sein. 
\endverse

\beginchorus
\[Dm]Ja dann säen, wir gemeinsam,\[Gm] backen unser eigenes Brot, 
\[C]trinken Wein aus vollen Schläuchen, \[F]tanzen bis ins \[A]Morgenrot. 
Bau'n \[Dm]noch ein Haus aus Kieselsteinen,\[Gm] pflanzen auch noch Majoran 
\[C]und du singst zu den Akkorden, \[F]ich spiel Geige \[A]was ich kann. 
\[Dm]Und das Holz im Ofen knistert,\[Gm] wenn du aus der Wanne steigst, 
\[C]der Kalender liegt im Feuer, \[F]wenn du mir den \[A]Nordstern zeigst...
{\nolyrics Schluss: \[Dm Gm C F A]}
\endchorus

\endsong


