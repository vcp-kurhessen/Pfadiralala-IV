\beginsong{Der Pfahl}[wuw={Luis Lach, übersetzt von Oss Kröher, 1968}, pfiii={56}, index={Sonnig begann es zu tagen}]

\markboth{\songtitle}{\songtitle}

\beginverse
\endverse

\centering\includegraphics[width=1\textwidth]{Noten/Lied018.pdf}	

\beginverse
\[Em]'Ach Siset, \[H7]noch ist es \[Em]nicht geschafft, an meiner Hand platzt die \[H7]Haut.
\[Em]Langsam auch \[H7]schwindet schon \[Em]meine Kraft, er ist zu \[H7]mächtig ge\[Em]baut.
Wird es uns \[H7]jemals ge\[Em]lingen? Siset, es fällt mir so \[H7]schwer!'
\[Em]'Wenn wir das \[H7]Lied nochmal \[Em]singen, geht es viel \[H7]besser. Komm \[Em]her!'
\endverse

\beginchorus 
\[Em]Ich drücke \[H7]hier und du ziehst \[Em]weg, so kriegen \[H7]wir den Pfahl vom \[Em]Fleck,
werden ihn \[Am]fällen, fällen, \[G]fällen, werfen ihn \[H7]morsch und faul zum \[Em]Dreck.
Erst wenn die \[H7]Eintracht uns be\[Em]wegt, haben wir \[H7]ihn bald umge\[Em]legt
und er wird \[Am]fallen, fallen, \[G]fallen, wenn sich ein \[H7]jeder von uns \[Em]regt!
\endchorus

\beginverse
^Der alte ^Siset sagt ^nichts mehr, böser Wind hat ihn ver^weht.
^Keiner weiß ^von seiner ^Heimkehr, keiner weiß, ^wie es ihm ^geht.
Alt-Siset ^sagte uns ^allen, hör es auch du, krieg es ^mit:
^Der morsche ^Pfahl wird schon ^fallen, wie es ge^schieht in dem ^Lied.
\endverse
\renewcommand{\everychorus}{\textnote{\bf Refrain (wdh.)}}
\beginchorus
\endchorus

\endsong

\beginscripture{}
Das Lied ist die Übersetzung von L'estaca von Luis Lach. Der Pfahl ist hier Sinnbild für den Staat. Das Lied ist zur Zeit der Diktatur in Katalonien bekannt geworden.
\endscripture

\begin{intersong}

\end{intersong}