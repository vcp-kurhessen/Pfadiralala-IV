\beginsong{Tutorial}[wuw={Künstler}, txt={Texter}, mel={Komponist}, jahr=2017, alb={Best of}, pfi=1, pfii=2, pfiii=3, bo=4, ju=5, index={Das ist die erste}]
% Die hinteren Informationen (außer Lied Titel) müssen nicht angegeben werden.

\interlude{Intro oder Zwischenspiel (ohne Text): \[Am C G]}

\beginverse\memorize
\[Am]Das ist die \[C]erste \[G]Strophe... 
\endverse

\beginchorus
\lrep \[Am]Und wir \[C]singen \[G]den Re\[Am]frain vier mal\rrep \rep{4}
\endchorus

\beginverse
^Zweite ^Strophe, Ak^korde wiederholt...
\endverse

\printchorus
% ... schreibt "Refrain (wdh.)"

\repchorus{5}
% ... schreibt "Refrain (5x)"



\endsong
\beginscripture{}
Hier werden spannende Zusatz-Informationen untergebracht, zum Beispiel, dass das Pfadiralala eines der tollsten Liederbücher ist!
\endscripture

\begin{intersong}

\end{intersong}