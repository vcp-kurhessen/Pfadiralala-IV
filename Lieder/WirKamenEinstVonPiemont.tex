\beginsong{Wir kamen einst von Piemont}[
    txt={cappi (Klaus Tränkle)}, 
    jahr={1952}, 
    wuw={französisches Volkslied}, 
    bo={398}, 
    gruen={98}, 
    kssiv={85}, 
    siru={288},
    tonspur={624}, 
]

\beginverse
\endverse
\includegraphics[draft=false, page=1]{Noten/WirKamenEinstVonPiemont.pdf}	

% \beginverse\memorize
% \lrep Wir \[E]kamen \[H7]einst von Pie\[E]mont und wollten \[A]weiter \[H7]nach Ly\[E]on \rrep
% Ach, im Beutel da herrschte \[H7]Leere,
% sans dessus dessous et sans devant derr\[E]ière
% Burschen war'n wir drei, doch \[H7]nur ein \[E]Sous!
% \endverse

\beginverse\memorize
\lrep Herr \[E]Wirt bring \[H7]uns ein Essen \[E]her! Der Magen \[A]ist so \[H7]lang schon \[E]leer. \rrep 
Hab' noch Fleisch von der alten \[H7]Mähre.
sans dessus dessous et sans devant derr\[E]ière
Ei, bring es her und \[H7]Wein da\[E]zu!
\lrep \[A]Sans devant derr\[E]ière sans des\[H7]sus dess\[E]ous.\rrep 
\endverse

\printchorus

\beginverse
\lrep Herr ^Wirt, wir ^woll'n nun weiter^zieh'n, das Essen ^war ge^wiss sehr ^schön. \rrep
Nehmt den Sous, wir hab'n nicht mehr, auf ^Ehre, 
sans dessus dessous et sans devant derr^ière
Wir aber stoben ^fort im ^Nu,
\lrep ^Sans devant derr^ière sans des^sus dess^ous.\rrep 
\endverse

\printchorus

\endsong
