\beginsong{O du Fröhliche}[
	mel={Sizilianisches Fischerlied}
    txt={Johannes Daniel Falk/Heinrich Holzschuher},
	txtjahr={1816 / 1826},
]


\beginverse\memorize
\[C]O \[F]du \[C]fröhliche, \[C]O \[F]du \[C]selige,
\[C]gna\[G]den\[D]brin\[G]gen\[Am]de \[G]Weih\[D]nachts\[G]zeit!
\[G]Welt ging \[C]ver\[Dm]loren, \[Am]Christ ward \[F]ge\[C]bo\[G]ren:
\[C]Freu\[F]e\[C], \[Am]freu\[F]e \[C]dich, \[Dm]O \[C]Chri\[G7]sten\[C]heit!
\endverse

\beginverse
^O ^du ^fröhliche, ^O ^du ^selige,
^gna^den^brin^gen^de ^Weih^nachts^zeit!
^Christ ist er^schie^nen, ^uns zu ^ver^süh^nen:
^Freu^e^, ^freu^e ^dich, ^O ^Chri^sten^heit!
\endverse

\beginverse
^O ^du ^fröhliche, ^O ^du ^selige,
^gna^den^brin^gen^de ^Weih^nachts^zeit!
^Himmlische ^Hee^re ^jauchzen ^dir ^Eh^re:
^Freu^e^, ^freu^e ^dich, ^O ^Chri^sten^heit!
\endverse

\endsong

\beginscripture{}
Die Melodie zu ''O du Fröhliche'' stammt von einem sizilianischen Fischerlied, das Johann Gottfried Herder 1788 von einer Italienreise mit nach Deutschland brachte. Zusammen mit den Texten von Johannes Daniel Falk (1. Strophe 1816) und Heinrich Holzschuher (2. + 3. Strophe, 1826) wurde daraus eines der beliebtesten Weihnachtslieder.
\endscripture