\beginsong{Das Recht auf Arbeitslosigkeit}[
	wuw={Götz Widmann}, 
	index={Ich bin bestimmt kein großer Anhänger}, 
	jahr={2001}, 
	alb={Götz Widmann},
]

\beginverse
Ich bin \[Dm]bestimmt kein großer Anhänger von \[A]Ethik und Moral,
wenn ein Politiker mich belügt, find ich das \[Dm]eigentlich normal,
mehr erwart ich nicht von ihm und solang \[A]er's nicht übertreibt,
ist mir schon klar, dass ihm im Grund auch gar nichts \[Dm]andres übrigbleibt.
Meinen \[C]Glauben an die Demokratie kann \[Dm]das nicht ernsthaft störn,
aber \[C]eines mag ich echt bei aller \[Dm]Liebe nicht mehr hörn.
Dieses Ge\[Gm]schwätz die ganze Zeit, egal welche Gelegenheit,
unser \[A]dringendstes Problem, das wär die Arbeitslosigkeit.
\endverse

\beginverse
Ich ^find das eine oberflächlich^e Betrachtungsweise.
Der Mensch, er ist der Mensch und nicht die ^rote Waldameise.
Die Welt ist nicht mehr so wie sie bei ^Adenauer war,
mal gründlicher betrachtet ist das ^alles nur blabla,
der ^Mensch ist nicht allein zum Funktio^nieren auf der Welt.
Der ^Mensch braucht nicht die Arbeit, der ^Mensch, er braucht nur Geld.
^ Es geht ihm nicht ums Stechuhr stechen,
^ sondern mehr ums Miete blechen.
\endverse

\beginchorus
\[Gm]Wenn man mir ein Recht gäb, ohne \[Dm]Arbeit gut zu leben,
\[Gm]Würde ich ein Recht auf Arbeit \[Dm]gar nicht mehr erstreben.
\[C] Maschinen schuften \[Dm]lassen\[C] und mit was besserem be\[Dm]fassen.
\endchorus

\beginverse
Die ^Sklavenhalterei, die Folter, ^die Inquisition,
die Pest, die Guillotine, das ^Scheibentelefon,
die Postkutsche, das Bleibenzin, das ^Wasser holen gehn,
das Reich, die DDR und ^das Schwarz-Weißfernsehn,
^ all das ham wir über^wunden,
nur ^gegen das Malochen hat noch keiner was er^funden.
Es mag noch ^tausend Jahre dauern, aber eins steht heut schon fest,
dass sich ^Arbeit eines Tages kollektiv vermeiden lässt.
\endverse

\beginchorus
\[Gm]Wenn wir uns ein kleines bisschen \[Dm]Mühe damit geben,
\[Gm]können wir ein Dasein ohne \[Dm]Arbeit noch erleben.
\[C] Maschinen schuften \[Dm]lassen\[C] und mit was besserem be\[Dm]fassen.
\endchorus

\beginverse
^Ach, was werden das für wunder^wunderschöne Zeiten.
Man spricht nicht mehr von Arbeitslosen,^ man spricht von Befreiten.
Die meisten kommn ihr ganzes Leben ^ohne Leistung klar,
manche nehmn sich nur so ab und ^zu ein freies Jahr.
Manche ^würden ihre Arbeit am liebsten auch noch ^rauchen,
denen ^gönn ich ihren Sonntag auf dem Golfplatz, wenn sie's ^brauchen.
Es ^wird so vieles kommen, wovon wir heute noch nichts ahnen.
Ich ^will hier kein Jahrtausend schon im vornherein verplanen.
\endverse

\beginchorus
\[Gm]Aber im Jahr 3000, was für \[Dm]Netze wir auch weben,
\[Gm]Arbeit wird es hoffentlich dann nur noch \[Dm]ausnahmsweise geben.
\[C] Maschinen schuften \[Dm]lassen\[C] und mit was besserem be\[Dm]fassen. \newline

Ich \[Gm]schlage deshalb vor, dass man nen \[Dm]Sonderfonds einrichtet
für den \[Gm]Teil des Volks, der freiwillig aufs \[Dm]Arbeiten verzichtet.
\[C] Maschinen schuften \[Dm]lassen\[C] und mit was besserem be\[Dm]fassen.
\endchorus

\endsong
