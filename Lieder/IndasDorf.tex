\beginsong{In das Dorf}[
    wuw={aus dem Russischen}, 
    bo={202}, 
    pfii={35}, 
    pfiii={16}, 
    gruen={56}, 
    kssiv={28}, 
    siru={129},
    tonspur={344}, 
]

\beginverse
\endverse
\includegraphics[draft=false, width=1\textwidth]{Noten/Lied053.pdf}	

\beginverse
\[H7]Bauern, in den \[Em]Stall die Schweine, \[H7]dem Zigeuner traue nie. \[Em]
\lrep \[Am]Nehmt die Wäsche \[Em]von der Leine, \[H7]rettet euer \[Em]Federvieh! \rrep
\endverse

\beginverse
^Kesselflicker, ^Scherenschleifer ^preisen ihre Künste laut. ^
\lrep ^Geiger spielen ^Korobuschka, ^schon tanzt Sven mit ^seiner Braut.\rrep
\endverse

\beginverse
^Mädchen, die auf ^Heirat warten, ^drängen sich an Maras Stand, ^
\lrep ^denn die Alte ^legt die Karten, ^liest die Zukunft ^aus der Hand.\rrep
\endverse

\beginverse
^Dudelsack quält ^ohne Pause, ^brauner Bär tanzt mit Gebrumm. ^
\lrep ^Spät geht frohes ^Volk nach Hause, ^dunkel liegt das ^Dorf und stumm.\rrep
\endverse

\endsong

\beginscripture{}
Das Lied zeichnet ein romantisiertes Bild der Bevölkerungsgruppe Sinti und Roma gepaart mit stereotypen Bildern: Fahrendes Volk, das in das Dorf kommt und für allerlei Trubel sorgt. Sinti und Roma wurden und werden in ihrer Geschichte verfolgt und diskriminiert, die romantisierte Darstellung im Lied verklärt das Unrecht und die Verfolgung, die dieser Minderheit über Jahrhunderte in Deutschland und Europa widerfahren ist. 

Die Melodie stammt vom russischen Motiv ''Korobeiniki'', das vor allem durch die ''Type A''-Musik aus Tetris für den Gameboy bekannt ist.
\endscripture
