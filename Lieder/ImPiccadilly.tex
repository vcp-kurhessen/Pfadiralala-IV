\beginsong{Im Piccadilly brennt noch Licht}[
    wuw={Eric Dühn (Die Alfredos), im Nerotherbund mündlich überliefert},
    jahr={1961},
]

\beginverse
% Im Picca\[A]dilly brennt noch \[E]Licht, doch alle Türen sind schon \[A]dicht.
% Tja, da tagt heute die Hautevo\[D]lee. Karten \[A]gibt es nur \[E]schwarz beim Por\[A]tier.
\endverse
\includegraphics[draft=false]{Noten/ImPiccadilly.pdf}

\beginverse\memorize
Und in der \[A]Ecke, ganz pri\[E]vat, 
sitzt eine Dame vom Se\[A]nat.
Das ist Herr Meier, heute im \[D]Rock 
aus dem \[A]Rathaus im \[E]siebzehnten \[A]Stock.
\endverse

\beginchorus
Heut' tanzt das \[A]Publikum den Tango \[E]anders 'rum. 
\lrep Bis morgens \[E]früh um drei tagt hier die \[A]andere Partei. \rrep
\endchorus

\beginverse
In dieser ^schwülen, heißen ^Luft 
mit ihrem süßen, schweren ^Duft,
trägt selbst der Mixer an der ^Bar, 
zur Feier des ^Tages Nar^zissen im ^Haar.
\endverse

\printchorus

\beginverse
Von der ^Kapelle hier im ^Saal, 
ist nur der Bassgeiger stinknor^mal. 
Doch all' die ander'n, tandara^dei, 
feiern ^heut' ihren ^siebzehnten ^Mai.
\endverse

\printchorus

\endsong

\beginscripture{}
Das im Lied besungene ''Piccadilly'' wurde am 30. Mai 1958 in Hamburg als eine der ersten Schwulenkneipe eröffnet. 
Der 17.5. wurde in Deutschland früher ironischerweise als Feiertag der Schwulen bezeichnet, weil der Paragraph 175 des StGB bis in die 1990er Jahre Homosexualität unter Strafe stellte.
\endscripture