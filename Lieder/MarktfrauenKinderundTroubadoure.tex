\beginsong{Marktfrauen, Kinder und Troubadoure}[
    wuw={rumpel (Andreas Barth), BdP Löwenherz, Marburg}, 
    jahr={1992}, 
    bo={270}, 
    pfii={39}, 
    pfiii={49}, 
    siru={202}, 
    tonspur={444}, 
    index={Scharlatane, eins will ich euch sagen},
]

\beginverse
\endverse
\includegraphics[draft=false, width=1\textwidth]{Noten/Lied063.pdf}	

\beginverse
\[Em]Sie war’n nicht \[D]lange im \[C]Ne\[D]bel ver\[Em]borgen,
sie hat nicht \[D]lange die \[C]Blind\[D]heit ge\[Em]quält.
\[Em]Sie haben \[D]Einfall um \[C]Ein\[D]fall ge\[Em]boren,
haben \[D]Gefahr und \[C]Frei\[D]heit ge\[Em]wählt.
\lrep Sie sind nicht \[D]feige, \[F]ängstlich und \[Am]klein.
\[Em]Marktfrauen, \[D]Kinder und \[C]Trou\[D]ba\[Em]doure
\[Em]tanzen in \[D]unsere \[C]Zeit \[D]hi\[Em]nein. \rrep
\endverse

\beginverse
^Trotzen der ^Macht, die mit ^Angst ^Menschen ^presste,
die allen ^Mut zu ^fra^gen ver^bannt.
^Feiern in^mitten der ^Ker^ker die ^Feste,
bauen ihr ^Zelt auf ver^bo^tenes ^Land.
\lrep Blumen er^blüh'n im ^grauen Ge^stein.
^Marktfrauen, ^Kinder und ^Trou^ba^doure,
^bunter ^Tanz in die ^Welt ^hi^nein. \rrep
\endverse

\endsong
