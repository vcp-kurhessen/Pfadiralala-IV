\beginsong{Presslufthammer Bernhard}[wuw={Torfrock}, jahr={1990}, alb={Torfrock oder Watt?}]

\beginverse\memorize
\[A]Die Leute erzählen, ich humpel wenn ich gehe.
Und die \[F#]Ohren wollen auch nicht mehr so richtig.
Inne \[E]Frühstückspause verschüttel' ich den Tee, 
aber für \[D]mich ist das alles nicht so \[A]wichtig.

Mit meinem Presslufthammer fühl ich mich verbunden
und bin \[F#]verknallt in Staub und Schutt
und ich \[E]ratter euch in \[A]nichtmal \[E]viereinhalb Stunden
ein \[A]Einfa\[E]milienhaus in \[A]Dutt. \[C] \[D] \[E]
\endverse


\beginchorus
\lrep \[A]Jeden Tag hol ich den \[D]Presslufthammer aus der 
\[H]Werkzeugkammer und dann \[E]mach ich Krach.
Das gibt \[D]keinen der seinen Hammer so \[A]gern hat. 
Man nennt mich \[A]Presslufthammer \[D]B-B-Bernhard. \[E] \[A]\rrep
Ratatazong, ratatazong, weg ist der Balkon, dong.
\endchorus

\beginverse
Mit meinem ^Chef versteh ich mich allerbest'.
Ich ^jammer ja auch nie rum,
von ^Lohnerhöhung oder Arbeitsgesetz
und er ^findet ich bin gar nicht so ^dumm.
\newpage
''Moin, moin, mein Bernhard mach hin und gut Holz'',
brüllt er ^mir manchmal ins Ohr
und dann ^krieg‚ ne Zigarre ^und denn ^bin ich ganz stolz
^und dann ^katter ich ihm ein ^vor. ^^^
\endverse

\printchorus

\beginverse
Nur ^einmal ist mein Chef in Not geraten,
da ^musste ich zu ihm nach Hause
und ^sollte ihm seinen Garten umgraben
und er ^sagte: ''Ich mach erst mal ^Pause...

Hier rumzustehen hat sowieso keinen Zweck,
Ich geh mal ^rüber und besuch' unseren Pastor.''
Und ^ich hab verstanden: ''^Knack hier ^alles weg,''
als ^er zu^rück kam lag sein Haus auf'm ^Laster. ^^^
\endverse

\printchorus

% \beginchorus
% Jeden Tach hol ich den... den.. also...
% womit man die Steine immer so kaputt mach'n tut, nich?
% ...also, das 'son Hammer mit harte Lu-Lu-Luft.
% Na iiissch ja egaaal Ich mu-mu-muss sssowieso wieder zu meine Arbeit.
% Nu hör' doch auf mit den rrrrrock'n'rrrrroll, da...
% das war doch almählich peinlich, hier oder, oder, oder, oder was?
% \endchorus

\endsong