\beginsong{Idas Sommerlied}[
	txt={Astrid Lindgren, Torsten Meiwald (Übersetzung)},
	mel={Georg Riedel},
	txtjahr={1973},
    wuw={Little Pink},
    alb={No Show},
	jahr={2023},
    % jahr={1979},
    % pfii={248},
    % pfiii={127},
    % kssiv={182},
    index={Glaub nicht, von allein würd' es Sommer},
]

\beginverse\memorize
Glaub \[D]nicht, von allein würd' es \[G]Sommer in \[A7]Garten und Wiese und \[D]Wald.
Den \[D]Sommer den muss jemand \[G]wecken, dann \[A7]blühen die Blumen schon \[D]bald.
Ich \[F#m]lasse die Blumen er\[Em]blühen, lass \[A7]sprießen das Gras und den \[F#7]Klee.
Ja, \[Hm]nun kann der Sommer be\[G]ginnen, denn \[A7]schmelzen ließ ich schon den \[D]Schnee.
\endverse


\beginverse
Ich ^lasse das Wasser schnell ^strömen und ^setze die Bäche in ^Gang,
lass ^Schwalben am Himmel jetzt ^fliegen und ^Mücken, den Schwalben zum ^Fang.
Ich ^schenke den Bäumen die ^Blätter und ^setze die Nester hin^ein.
Ich ^lasse den Himmel er^glühen am ^Abend mit rosigem ^Schein.
\endverse

\beginverse
Und ^Waldbeeren werde ich ^machen, ich ^finde die braucht jedes ^Kind,
und ^andere herrliche ^Sachen, die ^passend für Kinder jetzt ^sind.
Ich ^mache so lustige ^Stellen, grad ^richtig zum Spielen mit ^dir.
Da ^hüpf ich und renne und ^springe und ^spüre den Sommer in ^mir.
\endverse

\beginverse
Och ^smultron det gör jag åt ^barna, för ^det tycker jag dom kan ^få,
och ^andra små roliga ^saker som ^passar när barnen är ^små.
Och ^jag gör så roliga ^ställen, där ^barnen kan springa om^kring,
då ^blir barna fulla med ^sommar och ^bena blir fulla med ^spring.
Ich \[F#,]mache so lustige \[Em]Stellen, grad \[A7]richtig zum Spielen mit \[F#7]dir.
Da \[Hm]hüpf ich und renne und \[G]springe und \[A7]spüre den Sommer in \[D]mir.
\endverse

\endsong
